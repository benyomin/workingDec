% This is the Reed College LaTeX thesis template. Most of the work
% for the document class was done by Sam Noble (SN), as well as this
% template. Later comments etc. by Ben Salzberg (BTS). Additional
% restructuring and APA support by Jess Youngberg (JY).
% Your comments and suggestions are more than welcome; please email
% them to cus@reed.edu
%
% See http://web.reed.edu/cis/help/latex.html for help. There are a
% great bunch of help pages there, with notes on
% getting started, bibtex, etc. Go there and read it if you're not
% already familiar with LaTeX.
%
% Any line that starts with a percent symbol is a comment.
% They won't show up in the document, and are useful for notes
% to yourself and explaining commands.
% Commenting also removes a line from the document;
% very handy for troubleshooting problems. -BTS

% As far as I know, this follows the requirements laid out in
% the 2002-2003 Senior Handbook. Ask a librarian to check the
% document before binding. -SN

%%
%% Preamble
%%
% \documentclass{<something>} must begin each LaTeX document
\documentclass[12pt,twoside]{reedthesis}
% Packages are extensions to the basic LaTeX functions. Whatever you
% want to typeset, there is probably a package out there for it.
% Chemistry (chemtex), screenplays, you name it.
% Check out CTAN to see: http://www.ctan.org/
%%
\usepackage{graphicx,latexsym}
\usepackage{amsmath}
\usepackage{amssymb,amsthm}
\usepackage{longtable,booktabs,setspace}
\usepackage{chemarr} %% Useful for one reaction arrow, useless if you're not a chem major
\usepackage[hyphens]{url}
% Added by CII
\usepackage{hyperref}
\usepackage{lmodern}
% End of CII addition
\usepackage{rotating}

% Next line commented out by CII
%%% \usepackage{natbib}
% Comment out the natbib line above and uncomment the following two lines to use the new 
% biblatex-chicago style, for Chicago A. Also make some changes at the end where the 
% bibliography is included. 
%\usepackage{biblatex-chicago}
%\bibliography{thesis}


% Added by CII (Thanks, Hadley!)
% Use ref for internal links
\renewcommand{\hyperref}[2][???]{\autoref{#1}}
\def\chapterautorefname{Chapter}
\def\sectionautorefname{Section}
\def\subsectionautorefname{Subsection}
% End of CII addition

% Added by CII 
\usepackage{caption}
\captionsetup{width=5in}
% End of CII addition

% \usepackage{times} % other fonts are available like times, bookman, charter, palatino


% To pass between YAML and LaTeX the dollar signs are added by CII
\title{Untitled Demography Proposal}
\author{Brandon Payne}
% The month and year that you submit your FINAL draft TO THE LIBRARY (May or December)
\date{Sept. 2016}
\division{Ben-Gurion Research Institute}
\advisor{Dr.~Ben Herzog, advisor}
%If you have two advisors for some reason, you can use the following
% Uncommented out by CII
\altadvisor{committee member} 
% End of CII addition

%%% Remember to use the correct department!
\department{Israel Studies International MA Program}
% if you're writing a thesis in an interdisciplinary major,
% uncomment the line below and change the text as appropriate.
% check the Senior Handbook if unsure.
%\thedivisionof{The Established Interdisciplinary Committee for}
% if you want the approval page to say "Approved for the Committee",
% uncomment the next line
%\approvedforthe{Committee}

% Added by CII
%%% Copied from knitr
%% maxwidth is the original width if it's less than linewidth
%% otherwise use linewidth (to make sure the graphics do not exceed the margin)
\makeatletter
\def\maxwidth{ %
  \ifdim\Gin@nat@width>\linewidth
    \linewidth
  \else
    \Gin@nat@width
  \fi
}
\makeatother

\renewcommand{\contentsname}{Table of Contents}
% End of CII addition

\setlength{\parskip}{0pt}

% Added by CII

\providecommand{\tightlist}{%
  \setlength{\itemsep}{0pt}\setlength{\parskip}{0pt}}

\Acknowledgements{

}

\Dedication{

}

\Preface{

}

\Abstract{

}

	\usepackage[authordate, notes, language=english, backend=biber]{biblatex-chicago}
	\ExecuteBibliographyOptions{firstinits=false, isbn=false, url=false, doi=true, uniquename=init}
% End of CII addition
%%
%% End Preamble
%%
%

\begin{document}

% Everything below added by CII
      \maketitle
  
  \frontmatter % this stuff will be roman-numbered
  \pagestyle{empty} % this removes page numbers from the frontmatter

  
  
      \hypersetup{linkcolor=black}
    \setcounter{tocdepth}{1}
    \tableofcontents
  
      \listoftables
  
      \listoffigures
  
  
  
  \mainmatter % here the regular arabic numbering starts
  \pagestyle{fancyplain} % turns page numbering back on

  \chapter*{Introduction}\label{introduction}
  \addcontentsline{toc}{chapter}{Introduction}
  
  \section{Kinds of Russians}\label{kinds-of-russians}
  
  \begin{itemize}
  \tightlist
  \item
    Jews
  \item
    non-Jews
  \item
    those who immigrated to Israel
  \item
    those who immigrated to other countries 1. US 2. Germany and W. Europe
    3. elsewhere
  \item
    those who stayed in Russia
  \end{itemize}
  
  The term \emph{Russian} in common Israeli parlance is a synedoche
  referring to immigrants from the several states of the Former Soviet
  Union (FSU) who have arrived in several waves since 1989. An immigrant
  who came to Israel\footnote{Immigrant to Israel: m.s. \emph{oleh}, f.s.
    \emph{olah}, pl. \emph{olim}.} in 1992 or 2002 from Georgia or Latvia
  is a \emph{Russian}. Former Foreign Minister Moshe Sharett was born in
  the Russian Empire, in what is now Ukraine. He made aliyah in 1910. He
  may be called an Ashkenazi Jew, a member of the Second Aliyah, the New
  Yishuv and the Generation of the Founders, but he is not called a
  \emph{Russian}. The 167,000 Soviet olim between the Six-Day War and 1988
  form an edge-case, they may be called either Israeli or Russian
  depending on their perceived level of integration into Israeli society.
  This research deals with the place of post-1989 FSU olim in the
  ethnically stratified Israeli labor market. It attempts to determine if
  the Israeli labor market meets the objective criteria set by economists
  that would classify it as ethnically stratified. It then looks for data
  which could aid in placing the Russians within this system of
  stratification. Finally it seeks to determine their place within this
  system and any changes that have occurred over time to the system and
  their position within it.
  
  \chapter{Population of Israel}\label{sec:orgheadline3}
  
  At the end of 2015 the population of Israel included 6.2 million Jews,
  1.7 million Arabs and 357,500 others.\footnote{(Statistics 2015).} Among
  these others are 300,000 non-Jewish immigrants from the FSU.\footnote{(Cohen
    2009).} These new immigrants from the FSU entered into a labor market
  with an existing system of stratification, with European and American
  Jews and the top, African and Asian Jews in the middle and Arab citizens
  of Israel at the bottom.\footnote{(Haberfeld and Cohen 2007).}
  Historical migration patterns had put this stratified system in place,
  with pre-state immigration mostly coming from Central and Eastern
  Europe. These early immigrants established the pre-state institutions
  that became the foundations of the Israeli state. They were in an
  advantageous position to occupy positions of high status and high
  political, economic and social power. The lasting socioeconomic effects
  of the circumstances surrounding initial absorption conditions and
  spatial location have been noted.\footnote{(Bar-haim and Semyonov 2015,
    325--26).} Among the dimensions of social stratification are income,
  occupation, education, political power, standard of living and place of
  residence.\footnote{(Semyonov and Lewin-Epstein 1987, 8).} This paper
  specifically reviews the question of occupational stratification as
  measured by occupational prestige.
  
  \section{Arrival of the Russians}\label{sec:orgheadline2}
  
  After the founding of the state, a Mass Immigration (1948-1952) took
  place which more than doubled the population from 600,000 to almost 1.5
  million. Immigration then became moderate and sporadic from 1953 to
  1989,(Bar-haim and Semyonov 2015, 325) averaging around 50,000 per
  year.(Goldscheider 2002, 43) The first wave of FSU olim, from 1989 to
  1991 numbered 400,000. The second wave lasted until 1995 and brought
  300,000 more. Between 1995 and 2006 50,000 more arrived in
  Israel.(Israel Central Bureau of Statistics 2009) The conditions of FSU
  olim absorption, including the shift in government policy in 1989-1990
  to \emph{direct absoption} in which olim received an \emph{absorption
  basket} of grants which they then had discrection to spend have been
  discussed and compared with earlier policies.(Semyonov, Moshe;
  Lewin-Epstein 2011) One finding was that, used to full-employment and
  low-quality housing in the Soviet Union, FSU olim would share crowded,
  expensive accomodations in the center of Israel rather than one-family
  flats in peripheral areas.\footnote{(Evans 2011).} Recent interviews
  with FSU olim who were University students in Israel showed the
  persistence of a drive for higher education, despite its decreased
  monetary rewards in Israel as compared to the Soviet Union.\footnote{(Lerner,
    Rapoport, and Lomsky-Feder 2007).}
  
  \chapter{Labor Market Stratification Theory}\label{sec:orgheadline4}
  
  The theory of an ethnically stratified labor market was elaborated by
  Edna Bonacich. Different ethnic groups in the labor market have varying
  costs of labor due to differing resources and motives for participation
  in the labor market. Ethnic antagonism will result from this price
  differential as the higher-priced group attempts to maintain employment.
  The expensive labor group will attempt to resist displacement in the
  market through exclusion or a caste arrangement.\footnote{(Bonacich
    1972).} Professors Semyonov and Lewin-Epstein have applied the theory
  to the Israeli case in several works, famously in the book \emph{Hewers
  of Wood and Drawer of Water} in which they found non-citizen Arabs who
  entered the labor market after the Six-Day war to have formed a
  lower-stratum beneath the three afore-mentioned groups.\footnote{(Semyonov
    and Lewin-Epstein 1987).} Recently Bar-Chaim and Semyonov identified
  ten geo cultural groups in the Israeli labor market based on ethnic
  origin and immigrant generation. These groups are immigrants from
  1.Asia, 2.Africa, 3.Europe and America, 4.the Former Soviet Union,
  5.Ethiopia. Second generation 6.Asian, 7.African, 8.European or
  American, 9.third generation Israeli Jews and 10.Arabs. Their work
  produced interesting findings and quantified observations made by other
  observers. For instance, immigrants from the FSU ranked 3rd highly in
  education, with 13.64 years of formal schooling to the 15.22 completed
  by the Israeli-born children of European and American immigrants and the
  14.58 years of European and American immigrants themselves. They had
  more than 2 and a half more years of average formal schooling than the
  Asian immigrant group's 11.02, and yet their mean ISEI occupational
  prestige was 43.84, only slightly above the Asian group's
  43.81.(Bar-haim and Semyonov 2015, 330)
  
  This research sought to deal specifically with the case of
  stratification in occupation. All occupations are ranked on a
  one-hundred point scale of occupational prestige. Respondents then are
  asked their ethnicity and occupation. From this data the range of
  occupations held by an ethnicity can be described and compared to that
  held by another group.
  
  This work discusses three refinements to Bonacich's Ethnic Antagonism.
  In the \emph{succession model}, each new group of immigrants takes the
  least-prestigious and poorest paid occupations in the hierarchy pushing
  existing groups up the ladder. They however prefer the \emph{queuing
  model} by which a surplus of lowest-status ethnic group members,
  competing for a limi
  
  \chapter{Methodology}\label{sec:orgheadline7}
  
  Many proxies are available for social position within a stratified
  society. These include individual or household income, household wealth,
  ownership of durable goods, access to education, health care and
  political power. Because of limited access to long-term, stable
  employment in large firms or the public sector, immigrants may turn to
  self-employment.(Semyonov and Barck de Raijman 1994, 376) Thus, there is
  a need to compare groups on the basis of total individual earnings,
  including self-employment income, and not on salary data alone.(Plaut
  2014) Differing labor-force participation rates by gender among the
  ethnic groups can yield differing household incomes from similar
  individual incomes. Some have sought to instead measure a household's
  standard of living by its consumption, rather than its income.(Lach
  2007, 579)
  
  \section{Occupational Prestige}\label{sec:orgheadline5}
  
  I sought to focus on the question of occupational prestige among FSU
  olim. The position of ethnic groups within the labor market is shown by
  their mean status and standard deviation on Andrea Tyree's 100 point
  scale for occupational prestige in Israel.\footnote{(Tyree 1981).} This
  scale has been employed in several papers.\footnote{(Semyonov and
    Lewin-Epstein 1987; Semyonov and Barck de Raijman 1994; Semyonov and
    Lewin-Epstein 1991).} These would later be converted to the
  International Socio-Economic Index of occupational status
  (ISEI).\footnote{(Ganzeboom, De Graaf, and Treiman 1992; Bar-haim and
    Semyonov 2015).} Their spread throughout the occupational structure is
  represented by an Index of concentration, with one representing a
  perfect spread among all occupations of all prestige levels. Changes
  over time have been studied by asking immigrants their current
  occupation, and their occupation of five years previous. A Socioeconomic
  Status gap between occupational status in 1983 and 1978 was then
  calculated. Israeli men gained in average status, moving from a mean
  occupational prestige on Tyree's scale of 44.5 to 48.0. Western European
  and North American immigrants suffered a drop in status from 59.1 to
  57.4 that was less than half the decrease (55.0 to 50.1) felt by
  immigrants from Latin America.\footnote{(Semyonov and Barck de Raijman
    1994, 383).}
  
  \section{Factors Affecting Income}\label{sec:orgheadline6}
  
  Generally those who are older and live in cities earn a higher income
  than young people in rural areas. Sixty-six percent of Israeli-born sons
  of European and American immigrants live in Metropolitan Areas, their
  average age in 45.22. The FSU immigrant population is more than a year
  older (46.6 years) and only forty-two percent of them live in
  Metropolitan Areas. How do these factors affect the 70\% more average
  income from salary (14,819 NIS/month to 8,646) received by the
  native-born group? How much of it is due to the previously discussed one
  and a half year gap in formal schooling between groups, and how much to
  occupational prestige?(Bar-haim and Semyonov 2015, 330) Semyonov and
  Lewin-Epstein have defined a process of \emph{Decomposing Income
  Differentials} in which multiple regression is used to determine the
  contribution made to income by the components of the model, and the
  residual which is due to ethnicity.
  
  \chapter{Review of Literature -}\label{review-of-literature--}
  
  \section{\texorpdfstring{\#\# \url{File:Lit} Review
  3\{\#litReview3\}Characteristics of the FSU
  pop}{\#\# File:Lit Review 3\{\#litReview3\}Characteristics of the FSU pop}}\label{filelit-review-3litreview3characteristics-of-the-fsu-pop}
  
  \section{Theories of Immigration}\label{theories-of-immigration}
  
  \section{Economic Stuff}\label{economic-stuff}
  
  \section{Policy of Absorbtion}\label{policy-of-absorbtion}
  
  \subsection{\texorpdfstring{``Israel as a Returning
  Diaspora''}{Israel as a Returning Diaspora}}\label{israel-as-a-returning-diaspora}
  
  nice recent paper with good overviews
  \href{file:///Users/AbuDavid/Documents/ThesisDocs/Amit,\%20Semyonov/2006/Amit,\%20Semyonov\%20-\%20Metropolis\%20world\%20bulletin\%20-\%202006.pdf}{Amit-Returning}
  
  \subsection{\texorpdfstring{``Ethnic Stratification in
  Israel''}{Ethnic Stratification in Israel}}\label{ethnic-stratification-in-israel}
  
  There was good stuff in Bar-Chaim
  \href{file:///Users/AbuDavid/Documents/ThesisDocs/Bar-haim,\%20Semyonov/2015/Bar-haim,\%20Semyonov\%20-\%20The\%20International\%20Handbook\%20of\%20the\%20Demography\%20of\%20Race\%20and\%20Ethnicity\%20-\%202015.pdf}{Bar-Chaim}
  timing: 1989 collapse of FSU -\textgreater{} influx of immigrants to
  Israel.
  
  \subsection{\texorpdfstring{``Migration Patterns to and from
  Israel''}{Migration Patterns to and from Israel}}\label{migration-patterns-to-and-from-israel}
  
  \subsection{\texorpdfstring{``Self-Selection in moving to US or
  IL''}{Self-Selection in moving to US or IL}}\label{self-selection-in-moving-to-us-or-il}
  
  \begin{itemize}
  \tightlist
  \item
    autocite:Cohen2007a
    \href{file:///Users/AbuDavid/Documents/ThesisDocs/Cohen,\%20Haberfeld/2007/Cohen,\%20Haberfeld\%20-\%20SELF-SELECTION\%20AND\%20EARNINGS\%20ASSIMILATION\%20IMMIGRANTS\%20FROM\%20THE\%20FORMER\%20SOVIET\%20UNION\%20IN\%20ISRAEL\%20AND\%20THE\%20UNITED\%20STATES\%20-\%2020.pdf}{cohen
    Self Selection} different destination countries have different returns
    to skills -p.650
  \end{itemize}
  
  US granted refugee status to J. from FSU until 1989 - what about exit
  visa from Soviet Union?
  
  after 1989 - family reunification visa to USA, but didn't other article
  say this was earlier?
  
  skilled immigrants go to countries with high returns to skills,
  low-skilled (less skilled) prefer social safety net countries
  
  compare rate of earnings growth for FSU immigrants v. US natives and IL
  natives of similar demographics
  
  \begin{enumerate}
  \def\labelenumi{\arabic{enumi}.}
  \item
    2 asumptions
  
    \begin{enumerate}
    \def\labelenumii{\arabic{enumii}.}
    \tightlist
    \item
      skills are equally transferable
  
      \begin{itemize}
      \tightlist
      \item
        non-native speaker
      \item
        advanced economies in IL and US
      \end{itemize}
    \item
      FSU immigrants are treated equally by US and IL labor markets
    \end{enumerate}
  \item
    2 waves of immigration = 1.8 Million J. + family members
  
    \begin{enumerate}
    \def\labelenumii{\arabic{enumii}.}
    \tightlist
    \item
      1968- early 80s 350k
    \item
      late
    \end{enumerate}
  \end{enumerate}
  
  \subsection{TODO Cohen 2009 - Migration Patterns to and from
  Israel}\label{todo-cohen-2009---migration-patterns-to-and-from-israel}
  
  autocite:cohen2009 has exported highlights (Cohen 2009, 115)
  \href{file:///Users/AbuDavid/Documents/ThesisDocs/Cohen/2009/Cohen\%20-\%20Migration\%20Patterns\%20to\%20and\%20from\%20Israel\%20-\%202009.pdf}{The
  pdf} these notes are in the file notes.org
  
  \begin{enumerate}
  \def\labelenumi{\arabic{enumi}.}
  \item
    success of Z.
  
    \begin{enumerate}
    \def\labelenumii{\arabic{enumii}.}
    \tightlist
    \item
      \% of world J. in IL
    \item
      \% J. v. A. in IL
    \item
      pop. growth rate of J. in IL
    \end{enumerate}
  \item
    p. 119 - 350k emigrants from IL, 1/2 in US
  
    CBS \# of 480k includes IL Arabs, approx. 100k 2003 ministry of
    absorbtion said 700k
  
    \begin{itemize}
    \tightlist
    \item
      didn't account for mortality abroad
    \end{itemize}
  \item
    rater of return migration are higher than most other sending countries
  
    \begin{itemize}
    \tightlist
    \item
      not every IL who abroad for 1 yr will never come backend
    \end{itemize}
  \item
    what is a good indicator for skills?
  
    \begin{itemize}
    \tightlist
    \item
      emigrants return better educated than when they left
    \item
      but those w/ higher income stay in US, lower return
    \end{itemize}
  \item
    FSU !!!
  
    \begin{enumerate}
    \def\labelenumii{\arabic{enumii}.}
    \tightlist
    \item
      1st wave 400k 1989-1991
  
      \begin{itemize}
      \tightlist
      \item
        400k immigrants to IL
      \item
        high schooling - 14 yrs
      \end{itemize}
    \item
      then, up to 2000 (after 1992)
  
      \begin{itemize}
      \tightlist
      \item
        60k to 80k annually
      \item
        only 13 yrs schooling
      \end{itemize}
    \item
      conclusion
  
      \begin{itemize}
      \tightlist
      \item
        in post 1991, higher edu. seek USA CANADA recently Germany
  
        \begin{itemize}
        \tightlist
        \item
          in 2002 22k FSU J. \textgreater{} IL or US
        \end{itemize}
      \end{itemize}
    \end{enumerate}
  \item
    impact of second intifada
  
    decline since 1999 -18k in 2007 -13k 2008 from 77k in 1999
  
    increase in \emph{annual emigration} (emigrants minus returnees)
  \end{enumerate}
  
  Conclusions \{.unnumbered\}
  
  The existing research has extracted estimable conclusions from the
  datasets available from the Israeli Central Bureau of Statistics - the
  2009 Income Survey and the Immigrants Survey. It has shown that due to
  the circumstances associated with the massive influx of FSU olim into
  the job market that they have still been unable to fully convert their
  human-capital into economic success in Israel.
  
  \chapter*{Appendix 1: Data Sets}\label{appendix-1-data-sets}
  \addcontentsline{toc}{chapter}{Appendix 1: Data Sets}
  
  \section{List of data sets obtained from the Central Bureau of
  Statistics for this
  project.}\label{list-of-data-sets-obtained-from-the-central-bureau-of-statistics-for-this-project.}
  
  These files are from four surveys for various years:
  
  \begin{enumerate}
  \def\labelenumi{\arabic{enumi}.}
  \tightlist
  \item
    Household Expenditure Survey
  \item
    Income Survey
  \item
    Labour Force Survey
  \item
    Social Survey
  \end{enumerate}
  
  \begin{longtable}[]{@{}lllll@{}}
  \toprule
  Census File & Year & Contents & Household & Individual\tabularnewline
  \midrule
  \endhead
  f210 & 2008 & Census Public Use File & x &\tabularnewline
  f456 & 2014 & HES & x & x\tabularnewline
  f457 & 2013 & HES & x & x\tabularnewline
  f458 & 2012 & HES & x & x\tabularnewline
  f459 & 2011 & HES & x & x\tabularnewline
  f467 & 2005 & HES & x & x\tabularnewline
  f468 & 2006 & HES & x & x\tabularnewline
  f469 & 2007 & HES & x & x\tabularnewline
  f472 & 2009 & HES & x & x\tabularnewline
  f598 & 2011 & HES & x & x\tabularnewline
  f599 & 2010 & HES & x & x\tabularnewline
  f606 & 2006 & Income Survey - Household & x &\tabularnewline
  f607 & 2007 & Income Survey - Household & x &\tabularnewline
  f608 & 2008 & Income Survey - Household & x &\tabularnewline
  f609 & 2009 & Income Survey - Household & x &\tabularnewline
  f787 & 2008 & Israel Social Survey & x &\tabularnewline
  f798puf & 2014 & Israel Social Survey & x &\tabularnewline
  f820puf & 2006 & Labour Force Survey & x &\tabularnewline
  f821puf & 2007 & Labour Force Survey & x &\tabularnewline
  f822puf & 2008 & Labour Force Survey & x &\tabularnewline
  f823pufn & 2009 & Labour Force Survey & x &\tabularnewline
  f823pufy & 2009 & Labour Force Survey & x &\tabularnewline
  f824puf & 2010 & Labour Force Survey & x &\tabularnewline
  f825puf & 2011 & Labour Force Survey & x &\tabularnewline
  \bottomrule
  \end{longtable}
  
  \nocite{*} \backmatter
  
  \chapter{References}\label{references}
  
  \noindent
  
  \setlength{\parindent}{-0.20in} \setlength{\leftskip}{0.20in}
  \setlength{\parskip}{8pt}
  
  \hypertarget{refs}{}
  \hypertarget{ref-Bar-Chaim2015}{}
  Bar-haim, Eyal, and Moshe Semyonov. 2015. ``Ethnic Stratifcation in
  Israel.'' \emph{The International Handbook of the Demography of Race and
  Ethnicity} 4: 323--37.
  doi:\href{https://doi.org/10.1007/978-90-481-8891-8}{10.1007/978-90-481-8891-8}.
  
  \hypertarget{ref-Bonacich1972}{}
  Bonacich, Edna. 1972. ``A Theory of Ethnic Antagonism: The Split Labor
  Market.'' \emph{American Sociological Review} 37 (5). {[}American
  Sociological Association, Sage Publications, Inc.{]}: 547--59.
  \url{http://www.jstor.org/stable/2093450}.
  
  \hypertarget{ref-cohen2009}{}
  Cohen, Yinon. 2009. ``Migration Patterns to and from Israel.''
  \emph{Contemporary Jewry} 29 (2: Jewish Population Studies): 115--25.
  doi:\href{https://doi.org/10.1007/s}{10.1007/s}.
  
  \hypertarget{ref-Over2015}{}
  Evans, Matt. 2011. ``Population Dispersal Policy and the 1990s
  Immigration Wave.'' \emph{Israel Studies} 20 (1): 104--28.
  
  \hypertarget{ref-Ganzeboom1992}{}
  Ganzeboom, Harry B G, Paul M. De Graaf, and Donald J. Treiman. 1992. ``A
  standard international socio-economic index of occupational status.''
  \emph{Social Science Research} 21 (1): 1--56.
  doi:\href{https://doi.org/10.1016/0049-089X(92)90017-B}{10.1016/0049-089X(92)90017-B}.
  
  \hypertarget{ref-goldscheider2001israel}{}
  Goldscheider, Calvin. 2002. \emph{Israel's Changing Society: Population,
  Ethnicity, and Development}. Boulder, Colorado: Westview Press.
  
  \hypertarget{ref-Haberfeld2007}{}
  Haberfeld, Yitchak, and Yinon Cohen. 2007. ``Gender, ethnic, and
  national earnings gaps in Israel: The role of rising inequality.''
  \emph{Social Science Research} 36 (2): 654--72.
  doi:\href{https://doi.org/10.1016/j.ssresearch.2006.02.001}{10.1016/j.ssresearch.2006.02.001}.
  
  \hypertarget{ref-Israel2008}{}
  Israel Central Bureau of Statistics. 2009. \emph{Israel in statistics,
  1948-2007}. Jerusalem: Israel Central Bureau of Statistics.
  \href{http://www.cbs.gov.il/statistical/statistical60\%7B/_\%7Deng.pdf}{http://www.cbs.gov.il/statistical/statistical60\{\textbackslash{}\_\}eng.pdf}.
  
  \hypertarget{ref-Lach2007}{}
  Lach, Saul. 2007. ``Immigration and Prices.'' \emph{Journal of Political
  Economy} 115 (4): 548--87.
  doi:\href{https://doi.org/10.1086/521529}{10.1086/521529}.
  
  \hypertarget{ref-Lerner2007}{}
  Lerner, Julia, Tamar Rapoport, and Edna Lomsky-Feder. 2007. ``The ethnic
  script in action: The regrounding of Russian Jewish immigrants in
  Israel.'' \emph{Ethos} 35 (2): 168--95.
  doi:\href{https://doi.org/10.1525/ETH.2007.35.2.168.THE}{10.1525/ETH.2007.35.2.168.THE}.
  
  \hypertarget{ref-Plaut2014a}{}
  Plaut, Steven. 2014. ``The Myth of Ethnic Inequality in Israel.''
  \emph{Middle East Quarterly}, 1--12.
  \url{http://www.meforum.org/3839/israel-inequality}.
  
  \hypertarget{ref-Semyonov1994a}{}
  Semyonov, Moshe, and Rebeca Barck de Raijman. 1994. \emph{Modes of labor
  market incorporation and occupational cost among new immigrants to
  Israel}. Discussion Paper (Mekhon Goldah Meir Le-Hikre 'Avodah
  Ve-Hevrah). Tel Aviv: Golda Meir Institute.
  
  \hypertarget{ref-Semyonov1987}{}
  Semyonov, Moshe, and Noah Lewin-Epstein. 1987. \emph{Hewers of wood and
  drawers of water: noncitizen Arabs in the Israeli labor market}. Cornell
  International Industrial and Labor Relations Reports. Ithaca, NY: ILR
  Press.
  
  \hypertarget{ref-Semyonov1991}{}
  ---------. 1991. ``Suburban labor markets, urban labor markets, and
  gender inequality in earning.'' \emph{Sociological Quarterly} 32 (4):
  611--20.
  
  \hypertarget{ref-Semyonov2011b}{}
  Semyonov, Moshe; Lewin-Epstein, Noah. 2011. ``Wealth Inequality: Ethnic
  Disparities in Israeli Society.'' \emph{Social Forces} 89 (March):
  935--59.
  doi:\href{https://doi.org/10.1353/sof.2011.0006}{10.1353/sof.2011.0006}.
  
  \hypertarget{ref-stat2015}{}
  Statistics, Central Bureau of. 2015. ``Statistical Abstract of Israel.''
  In. Jerusalem: Central Bureau of Statistics.
  \url{http://www.cbs.gov.il/reader/shnaton/shnatone_new.htm?CYear=2015\&Vol=66\&CSubject=2}.
  
  \hypertarget{ref-Tyree1981}{}
  Tyree, Andrea. 1981. ``Occupational Socioeconomic Status, Ethnicity and
  Sex in Israel: Consideration in Scale Construction.'' \emph{Megamot} 27
  (1). Henrietta Szold Institute: 7--21.
  \url{http://www.jstor.org/stable/23655402}.


  % Index?

\end{document}

