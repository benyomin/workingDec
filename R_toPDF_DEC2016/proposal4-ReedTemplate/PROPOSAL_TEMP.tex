% This is the Reed College LaTeX thesis template. Most of the work
% for the document class was done by Sam Noble (SN), as well as this
% template. Later comments etc. by Ben Salzberg (BTS). Additional
% restructuring and APA support by Jess Youngberg (JY).
% Your comments and suggestions are more than welcome; please email
% them to cus@reed.edu
%
% See http://web.reed.edu/cis/help/latex.html for help. There are a
% great bunch of help pages there, with notes on
% getting started, bibtex, etc. Go there and read it if you're not
% already familiar with LaTeX.
%
% Any line that starts with a percent symbol is a comment.
% They won't show up in the document, and are useful for notes
% to yourself and explaining commands.
% Commenting also removes a line from the document;
% very handy for troubleshooting problems. -BTS

% As far as I know, this follows the requirements laid out in
% the 2002-2003 Senior Handbook. Ask a librarian to check the
% document before binding. -SN

%%
%% Preamble
%%
% \documentclass{<something>} must begin each LaTeX document
\documentclass[12pt,twoside]{reedthesis}
% Packages are extensions to the basic LaTeX functions. Whatever you
% want to typeset, there is probably a package out there for it.
% Chemistry (chemtex), screenplays, you name it.
% Check out CTAN to see: http://www.ctan.org/
%%
\usepackage{graphicx,latexsym}
\usepackage{amsmath}
\usepackage{amssymb,amsthm}
\usepackage{longtable,booktabs,setspace}
\usepackage{chemarr} %% Useful for one reaction arrow, useless if you're not a chem major
\usepackage[hyphens]{url}
% Added by CII
\usepackage{hyperref}
\usepackage{lmodern}
% End of CII addition
\usepackage{rotating}

% Next line commented out by CII
%%% \usepackage{natbib}
% Comment out the natbib line above and uncomment the following two lines to use the new 
% biblatex-chicago style, for Chicago A. Also make some changes at the end where the 
% bibliography is included. 
%\usepackage{biblatex-chicago}
%\bibliography{thesis}


% Added by CII (Thanks, Hadley!)
% Use ref for internal links
\renewcommand{\hyperref}[2][???]{\autoref{#1}}
\def\chapterautorefname{Chapter}
\def\sectionautorefname{Section}
\def\subsectionautorefname{Subsection}
% End of CII addition

% Added by CII 
\usepackage{caption}
\captionsetup{width=5in}
% End of CII addition

% \usepackage{times} % other fonts are available like times, bookman, charter, palatino


% To pass between YAML and LaTeX the dollar signs are added by CII
\title{Untitled Demography Proposal}
\author{Brandon Payne}
% The month and year that you submit your FINAL draft TO THE LIBRARY (May or December)
\date{Sept. 2016}
\division{Ben-Gurion Research Institute}
\advisor{Dr.~Ben Herzog}
%If you have two advisors for some reason, you can use the following
% Uncommented out by CII
% End of CII addition

%%% Remember to use the correct department!
\department{Israel Studies International MA Program}
% if you're writing a thesis in an interdisciplinary major,
% uncomment the line below and change the text as appropriate.
% check the Senior Handbook if unsure.
%\thedivisionof{The Established Interdisciplinary Committee for}
% if you want the approval page to say "Approved for the Committee",
% uncomment the next line
%\approvedforthe{Committee}

% Added by CII
%%% Copied from knitr
%% maxwidth is the original width if it's less than linewidth
%% otherwise use linewidth (to make sure the graphics do not exceed the margin)
\makeatletter
\def\maxwidth{ %
  \ifdim\Gin@nat@width>\linewidth
    \linewidth
  \else
    \Gin@nat@width
  \fi
}
\makeatother

\renewcommand{\contentsname}{Table of Contents}
% End of CII addition

\setlength{\parskip}{0pt}

% Added by CII

\providecommand{\tightlist}{%
  \setlength{\itemsep}{0pt}\setlength{\parskip}{0pt}}

\Acknowledgements{

}

\Dedication{

}

\Preface{
This is a very important contribution.
}

\Abstract{
The preface pretty much says it all. \par  Second paragraph of abstract
starts here.
}

% End of CII addition
%%
%% End Preamble
%%
%

\begin{document}

% Everything below added by CII
      \maketitle
  
  \frontmatter % this stuff will be roman-numbered
  \pagestyle{empty} % this removes page numbers from the frontmatter

  
      \begin{preface}
      This is a very important contribution.
    \end{preface}
  
      \hypersetup{linkcolor=black}
    \setcounter{tocdepth}{1}
    \tableofcontents
  
      \listoftables
  
      \listoffigures
  
      \begin{abstract}
      The preface pretty much says it all. \par  Second paragraph of abstract
      starts here.
    \end{abstract}
  
  
  \mainmatter % here the regular arabic numbering starts
  \pagestyle{fancyplain} % turns page numbering back on

  \chapter*{Introduction}\label{introduction}
  \addcontentsline{toc}{chapter}{Introduction}
  
  This proposal is about Russians, who moved to Israel, and ended up being
  counted by the Central Bureau of Statistics. What does the data
  collected say about them?
  
  \subsubsection{Section 1:}\label{section-1}
  
  I found this stuff in old books:
  
  \subsubsection{Section 2:}\label{section-2}
  
  Taken all together, I think this means, something.
  
  \section{Kinds of Russians}\label{kinds-of-russians}
  
  \begin{itemize}
  \tightlist
  \item
    Jews
  \item
    non-Jews
  \item
    those who immigrated to Israel
  \item
    those who immigrated to other countries 1. US 2. Germany and W. Europe
    3. elsewhere
  \item
    those who stayed in Russia
  \end{itemize}
  
  Some inline code here: \textgreater{} The \texttt{cos} of \(2 \pi\) is
  1. \#\# Including plots
  
  \begin{center}\includegraphics{PROPOSAL_TEMP_files/figure-latex/pressure-1} \end{center}
  
  ``\{r include\_reedtemplates\_2, include = FALSE\} \# This chunk ensures
  that the reedtemplates package is installed and loaded \# This
  reedtemplates package includes the template files for the thesis and
  also \# two functions used for labeling and referencing
  if(!require(devtools)) install.packages(``devtools'', repos =
  ``\url{http://cran.rstudio.com}'')
  
  if(!require(reedtemplates))\{ library(devtools)
  devtools::install\_github(``ismayc/reedtemplates'') \}
  library(reedtemplates) ```
  
  \chapter{Mathematics and Science}\label{math-sci}
  
  \section{Math}\label{math}
  
  \begin{align*}
  \mathrm{Unit.Sales}_{t} = \alpha 
      &+ \beta_{1}  \mathrm{Date}_{t} \\
      &+ \beta_{2}  \mathrm{Cable.TV}_{t}    \\
      &+ \beta_{3}  \mathrm{Int.Display}_{t} \\
      &+ \beta_{4}  \mathrm{Local.Radio}_{t} \\
      &+ \beta_{5}  \mathrm{Magazines}_{t}   \\
      &+ \beta_{6}  \mathrm{Network.TV}_{t}  \\
      &+ \beta_{7}  \mathrm{Spot.TV}_{t} + \epsilon
  \end{align*}
  
  In the \textbf{R} chunk below, we will load in a picture stored as
  \texttt{reed.jpg} in our main directory. We then give it the caption of
  ``Reed logo'', the label of ``reed'', and specify that this is a figure.
  Note again the use of the \texttt{results\ =\ "asis"} specification to
  automatically include and compile the \LaTeX~code.
  
  \begin{Shaded}
  \begin{Highlighting}[]
  \KeywordTok{label}\NormalTok{(}\DataTypeTok{path =} \StringTok{"figure/reed.jpg"}\NormalTok{, }\DataTypeTok{caption =} \StringTok{"Reed logo"}\NormalTok{, }
        \DataTypeTok{label =} \StringTok{"reed"}\NormalTok{, }\DataTypeTok{type =} \StringTok{"figure"}\NormalTok{)}
  \end{Highlighting}
  \end{Shaded}
  
  \begin{figure}[h!tbp]
  \centering
  \includegraphics[angle = 0,scale = 1]{figure/reed.jpg}
  \caption[Reed logo]{\normalsize{Reed logo}}
  \label{fig:reed}
  \end{figure}
  
  \clearpage 
  
  \section{Footnotes and Endnotes}\label{footnotes-and-endnotes}
  
  You might want to footnote something.\footnote{footnote text} The
  footnote will be in a smaller font and placed appropriately. Endnotes
  work in much the same way. More information can be found about both on
  the CUS site or feel free to reach out to
  \href{mailto:data@reed.edu}{\nolinkurl{data@reed.edu}}.
  
  \section{Bibliographies}\label{bibliographies}
  
  Of course you will need to cite things, and you will probably accumulate
  an armful of sources. There are a variety of tools available for
  creating a bibliography database (stored with the .bib extension). In
  addition to BibTeX suggested below, you may want to consider using the
  free and easy-to-use tool called Zotero. The Reed librarians have
  created Zotero documentation at
  \url{http://libguides.reed.edu/citation/zotero}. In addition, a tutorial
  is available from Middlebury College at
  \url{http://sites.middlebury.edu/zoteromiddlebury/}.
  
  \emph{R Markdown} uses \emph{pandoc} (\url{http://pandoc.org/}) to build
  its bibliographies. One nice caveat of this is that you won't have to do
  a second compile to load in references as standard \LaTeX~requires. To
  cite references in your thesis (after creating your bibliography
  database), place the reference name inside square brackets and precede
  it by the ``at'' symbol. For example, here's a reference to a book about
  worrying: (Molina \& Borkovec, 1994). This \texttt{Molina1994} entry
  appears in a file called \texttt{thesis.bib} in the \texttt{bib} folder.
  This bibliography database file was created by a program called BibTeX.
  You can call this file something else if you like (look at the YAML
  header in the main .Rmd file) and, by default, is to placed in the
  \texttt{bib} folder.
  
  For more information about BibTeX and bibliographies, see our CUS site
  (\url{http://web.reed.edu/cis/help/latex/index.html})\footnote{Reed~College
    (2007)}. There are three pages on this topic: \emph{bibtex} (which
  talks about using BibTeX, at
  \url{http://web.reed.edu/cis/help/latex/bibtex.html}),
  \emph{bibtexstyles} (about how to find and use the bibliography style
  that best suits your needs, at
  \url{http://web.reed.edu/cis/help/latex/bibtexstyles.html}) and
  \emph{bibman} (which covers how to make and maintain a bibliography by
  hand, without BibTeX, at
  \url{http://web.reed.edu/cis/help/latex/bibman.html}). The last page
  will not be useful unless you have only a few sources.
  
  If you look at the YAML header at the top of the main .Rmd file you can
  see that we can specify the style of the bibliography by referencing the
  appropriate csl file. You can download a variety of different style
  files at \url{https://www.zotero.org/styles}. Make sure to download the
  file into the csl folder.
  
  \paragraph{Tips for Bibliographies}\label{tips-for-bibliographies}
  
  \begin{itemize}
  \tightlist
  \item
    Like with thesis formatting, the sooner you start compiling your
    bibliography for something as large as thesis, the better. Typing in
    source after source is mind-numbing enough; do you really want to do
    it for hours on end in late April? Think of it as procrastination.
  \item
    The cite key (a citation's label) needs to be unique from the other
    entries.
  \item
    When you have more than one author or editor, you need to separate
    each author's name by the word ``and'' e.g.
    \texttt{Author\ =\ \{Noble,\ Sam\ and\ Youngberg,\ Jessica\},}.
  \item
    Bibliographies made using BibTeX (whether manually or using a manager)
    accept \LaTeX~markup, so you can italicize and add symbols as
    necessary.
  \item
    To force capitalization in an article title or where all lowercase is
    generally used, bracket the capital letter in curly braces.
  \item
    You can add a Reed Thesis citation\footnote{Noble (2002)} option. The
    best way to do this is to use the phdthesis type of citation, and use
    the optional ``type'' field to enter ``Reed thesis'' or
    ``Undergraduate thesis.''
  \end{itemize}
  
  \section{Anything else?}\label{anything-else}
  
  If you'd like to see examples of other things in this template, please
  contact the Data @ Reed team (email
  \href{mailto:data@reed.edu}{\nolinkurl{data@reed.edu}}) with your
  suggestions. We love to see people using \emph{R Markdown} for their
  theses, and are happy to help.
  
  \chapter*{Conclusion}\label{conclusion}
  \addcontentsline{toc}{chapter}{Conclusion}
  
  \setcounter{chapter}{4} \setcounter{section}{0}
  
  If we don't want Conclusion to have a chapter number next to it, we can
  add the \texttt{\{.unnumbered\}} attribute. This has an unintended
  consequence of the sections being labeled as 3.6 for example though
  instead of 4.1. The \LaTeX~commands immediately following the Conclusion
  declaration get things back on track.
  
  \subsubsection{More info}\label{more-info}
  
  And here's some other random info: the first paragraph after a chapter
  title or section head \emph{shouldn't be} indented, because indents are
  to tell the reader that you're starting a new paragraph. Since that's
  obvious after a chapter or section title, proper typesetting doesn't add
  an indent there.
  
  \appendix
  
  \chapter{The First Appendix}\label{the-first-appendix}
  
  This first appendix includes all of the R chunks of code that were
  hidden throughout the document (using the \texttt{include\ =\ FALSE}
  chunk tag) to help with readibility and/or setup.
  
  \subsubsection{In the main Rmd file:}\label{in-the-main-rmd-file}
  
  \begin{Shaded}
  \begin{Highlighting}[]
  \CommentTok{# This chunk ensures that the reedtemplates package is}
  \CommentTok{# installed and loaded. This reedtemplates package includes}
  \CommentTok{# the template files for the thesis and also two functions}
  \CommentTok{# used for labeling and referencing}
  \NormalTok{if(!}\KeywordTok{require}\NormalTok{(devtools))}
    \KeywordTok{install.packages}\NormalTok{(}\StringTok{"devtools"}\NormalTok{, }\DataTypeTok{repos =} \StringTok{"http://cran.rstudio.com"}\NormalTok{)}
  \NormalTok{if(!}\KeywordTok{require}\NormalTok{(reedtemplates))\{}
    \KeywordTok{library}\NormalTok{(devtools)}
    \NormalTok{devtools::}\KeywordTok{install_github}\NormalTok{(}\StringTok{"ismayc/reedtemplates"}\NormalTok{)}
  \NormalTok{\}}
  \KeywordTok{library}\NormalTok{(reedtemplates)}
  \end{Highlighting}
  \end{Shaded}
  
  \subsubsection{\texorpdfstring{In
  \protect\hyperlink{ref_labels}{}:}{In :}}\label{in}
  
  \chapter{The Second Appendix, for
  Fun}\label{the-second-appendix-for-fun}
  
  \backmatter
  
  \chapter{References}\label{references}
  
  \noindent
  
  \setlength{\parindent}{-0.20in} \setlength{\leftskip}{0.20in}
  \setlength{\parskip}{8pt}
  
  \hypertarget{refs}{}
  \hypertarget{ref-Molina1994}{}
  Molina, S. T., \& Borkovec, T. D. (1994). The Penn State worry
  questionnaire: Psychometric properties and associated characteristics.
  In G. C. L. Davey \& F. Tallis (Eds.), \emph{Worrying: Perspectives on
  theory, assessment and treatment} (pp. 265--283). New York: Wiley.
  
  \hypertarget{ref-noble2002}{}
  Noble, S. G. (2002). \emph{Turning images into simple line-art}
  (Undergraduate thesis). Reed College.
  
  \hypertarget{ref-reedweb2007}{}
  Reed~College. (2007, march). LaTeX your document. Retrieved from
  \url{http://web.reed.edu/cis/help/LaTeX/index.html}


  % Index?

\end{document}

